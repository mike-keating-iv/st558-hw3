% Options for packages loaded elsewhere
\PassOptionsToPackage{unicode}{hyperref}
\PassOptionsToPackage{hyphens}{url}
\PassOptionsToPackage{dvipsnames,svgnames,x11names}{xcolor}
%
\documentclass[
  letterpaper,
  DIV=11,
  numbers=noendperiod]{scrartcl}

\usepackage{amsmath,amssymb}
\usepackage{iftex}
\ifPDFTeX
  \usepackage[T1]{fontenc}
  \usepackage[utf8]{inputenc}
  \usepackage{textcomp} % provide euro and other symbols
\else % if luatex or xetex
  \usepackage{unicode-math}
  \defaultfontfeatures{Scale=MatchLowercase}
  \defaultfontfeatures[\rmfamily]{Ligatures=TeX,Scale=1}
\fi
\usepackage{lmodern}
\ifPDFTeX\else  
    % xetex/luatex font selection
\fi
% Use upquote if available, for straight quotes in verbatim environments
\IfFileExists{upquote.sty}{\usepackage{upquote}}{}
\IfFileExists{microtype.sty}{% use microtype if available
  \usepackage[]{microtype}
  \UseMicrotypeSet[protrusion]{basicmath} % disable protrusion for tt fonts
}{}
\makeatletter
\@ifundefined{KOMAClassName}{% if non-KOMA class
  \IfFileExists{parskip.sty}{%
    \usepackage{parskip}
  }{% else
    \setlength{\parindent}{0pt}
    \setlength{\parskip}{6pt plus 2pt minus 1pt}}
}{% if KOMA class
  \KOMAoptions{parskip=half}}
\makeatother
\usepackage{xcolor}
\setlength{\emergencystretch}{3em} % prevent overfull lines
\setcounter{secnumdepth}{-\maxdimen} % remove section numbering
% Make \paragraph and \subparagraph free-standing
\makeatletter
\ifx\paragraph\undefined\else
  \let\oldparagraph\paragraph
  \renewcommand{\paragraph}{
    \@ifstar
      \xxxParagraphStar
      \xxxParagraphNoStar
  }
  \newcommand{\xxxParagraphStar}[1]{\oldparagraph*{#1}\mbox{}}
  \newcommand{\xxxParagraphNoStar}[1]{\oldparagraph{#1}\mbox{}}
\fi
\ifx\subparagraph\undefined\else
  \let\oldsubparagraph\subparagraph
  \renewcommand{\subparagraph}{
    \@ifstar
      \xxxSubParagraphStar
      \xxxSubParagraphNoStar
  }
  \newcommand{\xxxSubParagraphStar}[1]{\oldsubparagraph*{#1}\mbox{}}
  \newcommand{\xxxSubParagraphNoStar}[1]{\oldsubparagraph{#1}\mbox{}}
\fi
\makeatother

\usepackage{color}
\usepackage{fancyvrb}
\newcommand{\VerbBar}{|}
\newcommand{\VERB}{\Verb[commandchars=\\\{\}]}
\DefineVerbatimEnvironment{Highlighting}{Verbatim}{commandchars=\\\{\}}
% Add ',fontsize=\small' for more characters per line
\usepackage{framed}
\definecolor{shadecolor}{RGB}{241,243,245}
\newenvironment{Shaded}{\begin{snugshade}}{\end{snugshade}}
\newcommand{\AlertTok}[1]{\textcolor[rgb]{0.68,0.00,0.00}{#1}}
\newcommand{\AnnotationTok}[1]{\textcolor[rgb]{0.37,0.37,0.37}{#1}}
\newcommand{\AttributeTok}[1]{\textcolor[rgb]{0.40,0.45,0.13}{#1}}
\newcommand{\BaseNTok}[1]{\textcolor[rgb]{0.68,0.00,0.00}{#1}}
\newcommand{\BuiltInTok}[1]{\textcolor[rgb]{0.00,0.23,0.31}{#1}}
\newcommand{\CharTok}[1]{\textcolor[rgb]{0.13,0.47,0.30}{#1}}
\newcommand{\CommentTok}[1]{\textcolor[rgb]{0.37,0.37,0.37}{#1}}
\newcommand{\CommentVarTok}[1]{\textcolor[rgb]{0.37,0.37,0.37}{\textit{#1}}}
\newcommand{\ConstantTok}[1]{\textcolor[rgb]{0.56,0.35,0.01}{#1}}
\newcommand{\ControlFlowTok}[1]{\textcolor[rgb]{0.00,0.23,0.31}{\textbf{#1}}}
\newcommand{\DataTypeTok}[1]{\textcolor[rgb]{0.68,0.00,0.00}{#1}}
\newcommand{\DecValTok}[1]{\textcolor[rgb]{0.68,0.00,0.00}{#1}}
\newcommand{\DocumentationTok}[1]{\textcolor[rgb]{0.37,0.37,0.37}{\textit{#1}}}
\newcommand{\ErrorTok}[1]{\textcolor[rgb]{0.68,0.00,0.00}{#1}}
\newcommand{\ExtensionTok}[1]{\textcolor[rgb]{0.00,0.23,0.31}{#1}}
\newcommand{\FloatTok}[1]{\textcolor[rgb]{0.68,0.00,0.00}{#1}}
\newcommand{\FunctionTok}[1]{\textcolor[rgb]{0.28,0.35,0.67}{#1}}
\newcommand{\ImportTok}[1]{\textcolor[rgb]{0.00,0.46,0.62}{#1}}
\newcommand{\InformationTok}[1]{\textcolor[rgb]{0.37,0.37,0.37}{#1}}
\newcommand{\KeywordTok}[1]{\textcolor[rgb]{0.00,0.23,0.31}{\textbf{#1}}}
\newcommand{\NormalTok}[1]{\textcolor[rgb]{0.00,0.23,0.31}{#1}}
\newcommand{\OperatorTok}[1]{\textcolor[rgb]{0.37,0.37,0.37}{#1}}
\newcommand{\OtherTok}[1]{\textcolor[rgb]{0.00,0.23,0.31}{#1}}
\newcommand{\PreprocessorTok}[1]{\textcolor[rgb]{0.68,0.00,0.00}{#1}}
\newcommand{\RegionMarkerTok}[1]{\textcolor[rgb]{0.00,0.23,0.31}{#1}}
\newcommand{\SpecialCharTok}[1]{\textcolor[rgb]{0.37,0.37,0.37}{#1}}
\newcommand{\SpecialStringTok}[1]{\textcolor[rgb]{0.13,0.47,0.30}{#1}}
\newcommand{\StringTok}[1]{\textcolor[rgb]{0.13,0.47,0.30}{#1}}
\newcommand{\VariableTok}[1]{\textcolor[rgb]{0.07,0.07,0.07}{#1}}
\newcommand{\VerbatimStringTok}[1]{\textcolor[rgb]{0.13,0.47,0.30}{#1}}
\newcommand{\WarningTok}[1]{\textcolor[rgb]{0.37,0.37,0.37}{\textit{#1}}}

\providecommand{\tightlist}{%
  \setlength{\itemsep}{0pt}\setlength{\parskip}{0pt}}\usepackage{longtable,booktabs,array}
\usepackage{calc} % for calculating minipage widths
% Correct order of tables after \paragraph or \subparagraph
\usepackage{etoolbox}
\makeatletter
\patchcmd\longtable{\par}{\if@noskipsec\mbox{}\fi\par}{}{}
\makeatother
% Allow footnotes in longtable head/foot
\IfFileExists{footnotehyper.sty}{\usepackage{footnotehyper}}{\usepackage{footnote}}
\makesavenoteenv{longtable}
\usepackage{graphicx}
\makeatletter
\newsavebox\pandoc@box
\newcommand*\pandocbounded[1]{% scales image to fit in text height/width
  \sbox\pandoc@box{#1}%
  \Gscale@div\@tempa{\textheight}{\dimexpr\ht\pandoc@box+\dp\pandoc@box\relax}%
  \Gscale@div\@tempb{\linewidth}{\wd\pandoc@box}%
  \ifdim\@tempb\p@<\@tempa\p@\let\@tempa\@tempb\fi% select the smaller of both
  \ifdim\@tempa\p@<\p@\scalebox{\@tempa}{\usebox\pandoc@box}%
  \else\usebox{\pandoc@box}%
  \fi%
}
% Set default figure placement to htbp
\def\fps@figure{htbp}
\makeatother

\KOMAoption{captions}{tableheading}
\makeatletter
\@ifpackageloaded{caption}{}{\usepackage{caption}}
\AtBeginDocument{%
\ifdefined\contentsname
  \renewcommand*\contentsname{Table of contents}
\else
  \newcommand\contentsname{Table of contents}
\fi
\ifdefined\listfigurename
  \renewcommand*\listfigurename{List of Figures}
\else
  \newcommand\listfigurename{List of Figures}
\fi
\ifdefined\listtablename
  \renewcommand*\listtablename{List of Tables}
\else
  \newcommand\listtablename{List of Tables}
\fi
\ifdefined\figurename
  \renewcommand*\figurename{Figure}
\else
  \newcommand\figurename{Figure}
\fi
\ifdefined\tablename
  \renewcommand*\tablename{Table}
\else
  \newcommand\tablename{Table}
\fi
}
\@ifpackageloaded{float}{}{\usepackage{float}}
\floatstyle{ruled}
\@ifundefined{c@chapter}{\newfloat{codelisting}{h}{lop}}{\newfloat{codelisting}{h}{lop}[chapter]}
\floatname{codelisting}{Listing}
\newcommand*\listoflistings{\listof{codelisting}{List of Listings}}
\makeatother
\makeatletter
\makeatother
\makeatletter
\@ifpackageloaded{caption}{}{\usepackage{caption}}
\@ifpackageloaded{subcaption}{}{\usepackage{subcaption}}
\makeatother

\usepackage{bookmark}

\IfFileExists{xurl.sty}{\usepackage{xurl}}{} % add URL line breaks if available
\urlstyle{same} % disable monospaced font for URLs
\hypersetup{
  pdftitle={Into The Tidyverse},
  pdfauthor={Mike Keating},
  colorlinks=true,
  linkcolor={blue},
  filecolor={Maroon},
  citecolor={Blue},
  urlcolor={Blue},
  pdfcreator={LaTeX via pandoc}}


\title{Into The Tidyverse}
\author{Mike Keating}
\date{}

\begin{document}
\maketitle


\subsection{Load Dependencies}\label{load-dependencies}

\begin{verbatim}
-- Attaching core tidyverse packages ------------------------ tidyverse 2.0.0 --
v dplyr     1.1.4     v readr     2.1.5
v forcats   1.0.0     v stringr   1.5.1
v ggplot2   3.5.2     v tibble    3.2.1
v lubridate 1.9.4     v tidyr     1.3.1
v purrr     1.0.4     
-- Conflicts ------------------------------------------ tidyverse_conflicts() --
x dplyr::filter() masks stats::filter()
x dplyr::lag()    masks stats::lag()
i Use the conflicted package (<http://conflicted.r-lib.org/>) to force all conflicts to become errors

Attaching package: 'palmerpenguins'


The following objects are masked from 'package:datasets':

    penguins, penguins_raw
\end{verbatim}

\subsection{Task 1}\label{task-1}

The data for this task is called data.txt and data2.txt. Download these
and put them in your data folder before answering the questions below.

We can use read\_csv functions to read in data. CSV is a comma-separated
file i.e.~any text file that uses commas as a delimiter to separate the
record values for each field. Therefore, to load data from a text file
we can use the read\_csv() method (or versions of it), even if the file
itself does not have a .csv extension.

In the following question, we are going to read in txt data. Part a has
us working with the data.txt file. Part b has you working with the
data2.txt file.

\subsubsection{Part a}\label{part-a}

We cannot use read\_csv() to read the data in data.txt because it uses a
comma (`,') as the delimiter (the separating character between values).
Instead, we must use read\_csv2(), which uses a semicolon (`;') as its
delimiter. This is helpful in reading data from European countries where
a comma may be used as a decimal point and not as a field separator.

\begin{Shaded}
\begin{Highlighting}[]
\NormalTok{data }\OtherTok{\textless{}{-}} \FunctionTok{read\_csv2}\NormalTok{(}\StringTok{\textquotesingle{}data/data.txt\textquotesingle{}}\NormalTok{)}
\end{Highlighting}
\end{Shaded}

\begin{verbatim}
i Using "','" as decimal and "'.'" as grouping mark. Use `read_delim()` for more control.
\end{verbatim}

\begin{verbatim}
Rows: 2 Columns: 3
-- Column specification --------------------------------------------------------
Delimiter: ";"
dbl (3): x, y, z

i Use `spec()` to retrieve the full column specification for this data.
i Specify the column types or set `show_col_types = FALSE` to quiet this message.
\end{verbatim}

\begin{Shaded}
\begin{Highlighting}[]
\NormalTok{data}
\end{Highlighting}
\end{Shaded}

\begin{verbatim}
# A tibble: 2 x 3
      x     y     z
  <dbl> <dbl> <dbl>
1     1     2     3
2     5     3     8
\end{verbatim}

\subsubsection{Part b}\label{part-b}

Read data delimited by ``6'' and assign factor, double, and character as
datatypes for each column.

\begin{Shaded}
\begin{Highlighting}[]
\NormalTok{data2 }\OtherTok{\textless{}{-}} \FunctionTok{read\_delim}\NormalTok{(}\StringTok{\textquotesingle{}data/data2.txt\textquotesingle{}}\NormalTok{, }\AttributeTok{delim =} \StringTok{\textquotesingle{}6\textquotesingle{}}\NormalTok{, }\AttributeTok{col\_types =} \StringTok{\textquotesingle{}fdc\textquotesingle{}}\NormalTok{)}
\NormalTok{data2}
\end{Highlighting}
\end{Shaded}

\begin{verbatim}
# A tibble: 3 x 3
  x         y z    
  <fct> <dbl> <chr>
1 1         2 3    
2 5         3 8    
3 7         4 2    
\end{verbatim}

\subsection{Task 2}\label{task-2}

The Portland Trailblazers are a National Basketball Association (NBA)
sports team. These data reflect the points scored by 9 Portland
Trailblazers players across the first 10 games of the 2021-2022 NBA
season. We are going to use these data to show off our data tidying
skills. The data we will be using for this task is called trailblazer,
and can be found on Moodle.

\subsubsection{Part a}\label{part-a-1}

Take a glimpse of the trailblazer data set to show that you have read in
the data correctly.

\begin{Shaded}
\begin{Highlighting}[]
\NormalTok{trailblazer }\OtherTok{\textless{}{-}} \FunctionTok{read\_csv}\NormalTok{(}\StringTok{\textquotesingle{}data/trailblazer.csv\textquotesingle{}}\NormalTok{, }\AttributeTok{show\_col\_types =} \ConstantTok{FALSE}\NormalTok{)}

\FunctionTok{glimpse}\NormalTok{(trailblazer)}
\end{Highlighting}
\end{Shaded}

\begin{verbatim}
Rows: 9
Columns: 11
$ Player      <chr> "Damian Lillard", "CJ McCollum", "Norman Powell", "Robert ~
$ Game1_Home  <dbl> 20, 24, 14, 8, 20, 5, 11, 2, 7
$ Game2_Home  <dbl> 19, 28, 16, 6, 9, 5, 18, 8, 11
$ Game3_Away  <dbl> 12, 20, NA, 0, 4, 8, 12, 5, 5
$ Game4_Home  <dbl> 20, 25, NA, 3, 17, 10, 17, 8, 9
$ Game5_Home  <dbl> 25, 14, 12, 9, 14, 9, 5, 3, 8
$ Game6_Away  <dbl> 14, 25, 14, 6, 13, 6, 19, 8, 8
$ Game7_Away  <dbl> 20, 20, 22, 0, 7, 0, 17, 7, 4
$ Game8_Away  <dbl> 26, 21, 23, 6, 6, 7, 15, 0, 0
$ Game9_Home  <dbl> 4, 27, 25, 19, 10, 0, 16, 2, 7
$ Game10_Home <dbl> 25, 7, 13, 12, 15, 6, 10, 4, 8
\end{verbatim}

\subsubsection{Part b}\label{part-b-1}

Pivot the data so that you have columns for Player, Game, Location,
Points. Display the first five rows of your data set. Save your new data
set as trailblazer\_longer. Your data set should contain 90 rows and 4
columns.

Let's get a glimpse at just the original column names:

\begin{Shaded}
\begin{Highlighting}[]
\FunctionTok{colnames}\NormalTok{(trailblazer)}
\end{Highlighting}
\end{Shaded}

\begin{verbatim}
 [1] "Player"      "Game1_Home"  "Game2_Home"  "Game3_Away"  "Game4_Home" 
 [6] "Game5_Home"  "Game6_Away"  "Game7_Away"  "Game8_Away"  "Game9_Home" 
[11] "Game10_Home"
\end{verbatim}

\begin{Shaded}
\begin{Highlighting}[]
\NormalTok{trailblazer\_longer }\OtherTok{\textless{}{-}}\NormalTok{ trailblazer }\SpecialCharTok{|\textgreater{}} 
  \FunctionTok{pivot\_longer}\NormalTok{(}\StringTok{"Game1\_Home"}\SpecialCharTok{:}\StringTok{"Game10\_Home"}\NormalTok{, }
               \AttributeTok{names\_to =} \FunctionTok{c}\NormalTok{(}\StringTok{"Game"}\NormalTok{, }\StringTok{"Location"}\NormalTok{), }
               \AttributeTok{names\_sep =} \StringTok{"\_"}\NormalTok{, }
               \AttributeTok{values\_to =} \StringTok{"Points"}\NormalTok{)}
\CommentTok{\# Show first 5 rows}
\FunctionTok{print}\NormalTok{(}\FunctionTok{head}\NormalTok{(trailblazer\_longer, }\DecValTok{5}\NormalTok{))}
\end{Highlighting}
\end{Shaded}

\begin{verbatim}
# A tibble: 5 x 4
  Player         Game  Location Points
  <chr>          <chr> <chr>     <dbl>
1 Damian Lillard Game1 Home         20
2 Damian Lillard Game2 Home         19
3 Damian Lillard Game3 Away         12
4 Damian Lillard Game4 Home         20
5 Damian Lillard Game5 Home         25
\end{verbatim}

\begin{Shaded}
\begin{Highlighting}[]
\CommentTok{\# And checking dimensions}
\FunctionTok{print}\NormalTok{(}\FunctionTok{dim}\NormalTok{(trailblazer\_longer))}
\end{Highlighting}
\end{Shaded}

\begin{verbatim}
[1] 90  4
\end{verbatim}

\subsubsection{Part c}\label{part-c}

Which players scored more, on average, when playing at home versus away?
Answer this question using a single pipeline

\begin{Shaded}
\begin{Highlighting}[]
\NormalTok{trailblazer\_home\_v\_away }\OtherTok{\textless{}{-}} 
\NormalTok{  trailblazer\_longer }\SpecialCharTok{|\textgreater{}} 
  \FunctionTok{pivot\_wider}\NormalTok{(}\AttributeTok{names\_from =} \StringTok{"Location"}\NormalTok{, }\AttributeTok{values\_from =} \StringTok{"Points"}\NormalTok{) }\SpecialCharTok{|\textgreater{}}
  \FunctionTok{group\_by}\NormalTok{(Player) }\SpecialCharTok{|\textgreater{}}
  \FunctionTok{mutate}\NormalTok{(}\AttributeTok{mean\_home =} \FunctionTok{mean}\NormalTok{(Home, }\AttributeTok{na.rm =} \ConstantTok{TRUE}\NormalTok{),}
         \AttributeTok{mean\_away =} \FunctionTok{mean}\NormalTok{(Away, }\AttributeTok{na.rm =} \ConstantTok{TRUE}\NormalTok{),}
         \AttributeTok{diff\_home\_away =}\NormalTok{ mean\_home }\SpecialCharTok{{-}}\NormalTok{ mean\_away) }\SpecialCharTok{|\textgreater{}}
  \FunctionTok{arrange}\NormalTok{(}\FunctionTok{desc}\NormalTok{(diff\_home\_away))}
  
\NormalTok{trailblazer\_home\_v\_away[}\DecValTok{8}\SpecialCharTok{:}\DecValTok{11}\NormalTok{,]}
\end{Highlighting}
\end{Shaded}

\begin{verbatim}
# A tibble: 4 x 7
# Groups:   Player [2]
  Player           Game    Home  Away mean_home mean_away diff_home_away
  <chr>            <chr>  <dbl> <dbl>     <dbl>     <dbl>          <dbl>
1 Jusuf Nurkic     Game8     NA     6      14.2       7.5           6.67
2 Jusuf Nurkic     Game9     10    NA      14.2       7.5           6.67
3 Jusuf Nurkic     Game10    15    NA      14.2       7.5           6.67
4 Robert Covington Game1      8    NA       9.5       3             6.5 
\end{verbatim}

The following players scored more points on average at home than away:

\begin{Shaded}
\begin{Highlighting}[]
\NormalTok{trailblazer\_home\_v\_away }\SpecialCharTok{|\textgreater{}}
  \FunctionTok{filter}\NormalTok{(diff\_home\_away }\SpecialCharTok{\textgreater{}} \DecValTok{0}\NormalTok{) }\SpecialCharTok{|\textgreater{}}
  \FunctionTok{distinct}\NormalTok{(Player) }\CommentTok{\# Distinct gives us unique values in our df}
\end{Highlighting}
\end{Shaded}

\begin{verbatim}
# A tibble: 5 x 1
# Groups:   Player [5]
  Player          
  <chr>           
1 Jusuf Nurkic    
2 Robert Covington
3 Nassir Little   
4 Damian Lillard  
5 Cody Zeller     
\end{verbatim}

\subsection{Task 3}\label{task-3}

For the next tasks, we are going to use the penguins data set in the
palmerpenguins package.

\subsubsection{Problem a}\label{problem-a}

\begin{Shaded}
\begin{Highlighting}[]
\CommentTok{\# Provided erroneous code.}
\NormalTok{penguins }\SpecialCharTok{|\textgreater{}}
  \FunctionTok{select}\NormalTok{(species, island, bill\_length\_mm) }\SpecialCharTok{|\textgreater{}}
  \FunctionTok{pivot\_wider}\NormalTok{(}
    \AttributeTok{names\_from =}\NormalTok{ island, }\AttributeTok{values\_from =}\NormalTok{ bill\_length\_mm}
\NormalTok{  )}
\end{Highlighting}
\end{Shaded}

\begin{verbatim}
Warning: Values from `bill_length_mm` are not uniquely identified; output will contain
list-cols.
* Use `values_fn = list` to suppress this warning.
* Use `values_fn = {summary_fun}` to summarise duplicates.
* Use the following dplyr code to identify duplicates.
  {data} |>
  dplyr::summarise(n = dplyr::n(), .by = c(species, island)) |>
  dplyr::filter(n > 1L)
\end{verbatim}

\begin{verbatim}
# A tibble: 3 x 4
  species   Torgersen  Biscoe      Dream     
  <fct>     <list>     <list>      <list>    
1 Adelie    <dbl [52]> <dbl [44]>  <dbl [56]>
2 Gentoo    <NULL>     <dbl [124]> <NULL>    
3 Chinstrap <NULL>     <NULL>      <dbl [68]>
\end{verbatim}

\begin{Shaded}
\begin{Highlighting}[]
\CommentTok{\# Using the suggested code to identify duplicates}
\NormalTok{penguins }\SpecialCharTok{|\textgreater{}} \FunctionTok{summarize}\NormalTok{(}\AttributeTok{n =} \FunctionTok{n}\NormalTok{(), }\AttributeTok{.by=}\FunctionTok{c}\NormalTok{(species,island)) }\SpecialCharTok{|\textgreater{}} \FunctionTok{filter}\NormalTok{(n }\SpecialCharTok{\textgreater{}} \DecValTok{1}\NormalTok{L)}
\end{Highlighting}
\end{Shaded}

\begin{verbatim}
# A tibble: 5 x 3
  species   island        n
  <fct>     <fct>     <int>
1 Adelie    Torgersen    52
2 Adelie    Biscoe       44
3 Adelie    Dream        56
4 Gentoo    Biscoe      124
5 Chinstrap Dream        68
\end{verbatim}

This error occurs because there are multiple values (

\paragraph{Explain what \textless NULL\textgreater, \textless dbl
{[}52{]}\textgreater, and \textless list\textgreater{}
mean:}\label{explain-what-null-dbl-52-and-list-mean}

\paragraph{\textless NULL\textgreater{}}\label{null}

In this case, \textless NULL\textgreater{} means that the given
combination of species and island do not exist. For example, there are
no penguins of the species ``Chinstrap'' on the island ``Torgersen''.

We can check this by attempting to filter by these values:

\begin{Shaded}
\begin{Highlighting}[]
\NormalTok{penguins }\SpecialCharTok{|\textgreater{}} \FunctionTok{select}\NormalTok{(species, island, bill\_length\_mm) }\SpecialCharTok{|\textgreater{}} \FunctionTok{filter}\NormalTok{(species }\SpecialCharTok{==} \StringTok{"Chinstrap"}\NormalTok{, island }\SpecialCharTok{==} \StringTok{"Torgersen"}\NormalTok{)}
\end{Highlighting}
\end{Shaded}

\begin{verbatim}
# A tibble: 0 x 3
# i 3 variables: species <fct>, island <fct>, bill_length_mm <dbl>
\end{verbatim}

We returned a df with 0 rows (no results matching our criteria!) and 3
columns.

\paragraph{\textless dbl {[}52{]}\textgreater{}}\label{dbl-52}

Each observation for bill length where species is ``Adelie'' and island
is ``Torgersen'' was combined into a single list of numbers (double).

\begin{Shaded}
\begin{Highlighting}[]
\NormalTok{penguins }\SpecialCharTok{|\textgreater{}} \FunctionTok{select}\NormalTok{(species, island, bill\_length\_mm) }\SpecialCharTok{|\textgreater{}} \FunctionTok{filter}\NormalTok{(species }\SpecialCharTok{==} \StringTok{"Adelie"}\NormalTok{, island }\SpecialCharTok{==} \StringTok{"Torgersen"}\NormalTok{) }\SpecialCharTok{|\textgreater{}} \FunctionTok{str}\NormalTok{()}
\end{Highlighting}
\end{Shaded}

\begin{verbatim}
tibble [52 x 3] (S3: tbl_df/tbl/data.frame)
 $ species       : Factor w/ 3 levels "Adelie","Chinstrap",..: 1 1 1 1 1 1 1 1 1 1 ...
 $ island        : Factor w/ 3 levels "Biscoe","Dream",..: 3 3 3 3 3 3 3 3 3 3 ...
 $ bill_length_mm: num [1:52] 39.1 39.5 40.3 NA 36.7 39.3 38.9 39.2 34.1 42 ...
\end{verbatim}

\paragraph{\textless list\textgreater{}}\label{list}

As mentioned above, the column was converted to the list datatype.

\subsubsection{Part b}\label{part-b-2}

Create the table our colleague was trying to create:

\begin{Shaded}
\begin{Highlighting}[]
\CommentTok{\# We coerced the columns into the datatype double to match the given output.}
\NormalTok{penguins }\SpecialCharTok{|\textgreater{}}
  \FunctionTok{select}\NormalTok{(species, island, bill\_length\_mm) }\SpecialCharTok{|\textgreater{}} 
  \FunctionTok{group\_by}\NormalTok{(species, island) }\SpecialCharTok{|\textgreater{}} 
  \FunctionTok{summarize}\NormalTok{(}\AttributeTok{n =} \FunctionTok{n}\NormalTok{()) }\SpecialCharTok{|\textgreater{}} 
  \FunctionTok{pivot\_wider}\NormalTok{(}\AttributeTok{names\_from =}\NormalTok{ island, }\AttributeTok{values\_from =}\NormalTok{ n, }
              \AttributeTok{values\_fill =} \DecValTok{0}\NormalTok{) }\SpecialCharTok{|\textgreater{}} \FunctionTok{mutate}\NormalTok{(}\AttributeTok{Biscoe =} \FunctionTok{as.double}\NormalTok{(Biscoe),}
                                         \AttributeTok{Dream =} \FunctionTok{as.double}\NormalTok{(Dream),}
                                         \AttributeTok{Torgersen =} \FunctionTok{as.double}\NormalTok{(Torgersen))}
\end{Highlighting}
\end{Shaded}

\begin{verbatim}
`summarise()` has grouped output by 'species'. You can override using the
`.groups` argument.
\end{verbatim}

\begin{verbatim}
# A tibble: 3 x 4
# Groups:   species [3]
  species   Biscoe Dream Torgersen
  <fct>      <dbl> <dbl>     <dbl>
1 Adelie        44    56        52
2 Chinstrap      0    68         0
3 Gentoo       124     0         0
\end{verbatim}

\subsection{Task 4}\label{task-4}

Fill in the missing values:

\begin{Shaded}
\begin{Highlighting}[]
\CommentTok{\# We will filter by na values first and then assign by species.}
\CommentTok{\# Missing value for Gentoo is 30}
\CommentTok{\# Missing value for Adelie is 26}

\NormalTok{penguins\_filled }\OtherTok{\textless{}{-}}\NormalTok{ penguins }\SpecialCharTok{|\textgreater{}} \FunctionTok{select}\NormalTok{(species, island, bill\_length\_mm) }\SpecialCharTok{|\textgreater{}} 
  \FunctionTok{mutate}\NormalTok{(}\AttributeTok{bill\_length\_mm =} \FunctionTok{case\_when}\NormalTok{(species }\SpecialCharTok{==} \StringTok{"Adelie"} \SpecialCharTok{\&} \FunctionTok{is.na}\NormalTok{(bill\_length\_mm) }\SpecialCharTok{\textasciitilde{}} \DecValTok{26}\NormalTok{,}
\NormalTok{            species }\SpecialCharTok{==} \StringTok{"Gentoo"} \SpecialCharTok{\&} \FunctionTok{is.na}\NormalTok{(bill\_length\_mm) }\SpecialCharTok{\textasciitilde{}} \DecValTok{30}\NormalTok{,}
            \AttributeTok{.default =}\NormalTok{ bill\_length\_mm}
\NormalTok{            ))}
\FunctionTok{head}\NormalTok{(penguins\_filled, }\DecValTok{10}\NormalTok{)}
\end{Highlighting}
\end{Shaded}

\begin{verbatim}
# A tibble: 10 x 3
   species island    bill_length_mm
   <fct>   <fct>              <dbl>
 1 Adelie  Torgersen           39.1
 2 Adelie  Torgersen           39.5
 3 Adelie  Torgersen           40.3
 4 Adelie  Torgersen           26  
 5 Adelie  Torgersen           36.7
 6 Adelie  Torgersen           39.3
 7 Adelie  Torgersen           38.9
 8 Adelie  Torgersen           39.2
 9 Adelie  Torgersen           34.1
10 Adelie  Torgersen           42  
\end{verbatim}




\end{document}
